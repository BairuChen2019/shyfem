
%------------------------------------------------------------------------
%
%    Copyright (C) 1985-2018  Georg Umgiesser
%
%    This file is part of SHYFEM.
%
%    SHYFEM is free software: you can redistribute it and/or modify
%    it under the terms of the GNU General Public License as published by
%    the Free Software Foundation, either version 3 of the License, or
%    (at your option) any later version.
%
%    SHYFEM is distributed in the hope that it will be useful,
%    but WITHOUT ANY WARRANTY; without even the implied warranty of
%    MERCHANTABILITY or FITNESS FOR A PARTICULAR PURPOSE. See the
%    GNU General Public License for more details.
%
%    You should have received a copy of the GNU General Public License
%    along with SHYFEM. Please see the file COPYING in the main directory.
%    If not, see <http://www.gnu.org/licenses/>.
%
%    Contributions to this file can be found below in the revision log.
%
%------------------------------------------------------------------------

%%%%%%%%%%%%%%%%%%%%%%%%%%%%%%%%%%%%%%%%%%%%%%%%%%%%%%%%%%
%%%%%%% basic heading for Latex %%%%%%%%%%%%%%%%%%%%%%%%%%
%%%%%%%%%%%%%%%%%%%%%%%%%%%%%%%%%%%%%%%%%%%%%%%%%%%%%%%%%%

\documentclass[12pt]{article}

\usepackage{a4}



%%%%%%%%%%%%%%%%%%%%%%%%%%%%%%%%%%%%%%%%%%%%%%%%%%%%%%%%%%
%%%%%%% user commands %%%%%%%%%%%%%%%%%%%%%%%%%%%%%%%%%%%%
%%%%%%%%%%%%%%%%%%%%%%%%%%%%%%%%%%%%%%%%%%%%%%%%%%%%%%%%%%

\newcommand{\vect}[2]   {       \mbox{$
					\left(  \begin{array}{c}
					#1 \\ #2
					\end{array}     \right)
				$}
			}
\newcommand{\paren}[1]  { \left( #1 \right) }

\newcommand{\nsz} {\normalsize}
\newcommand{\tdif}[1] {\frac{\partial #1}{\partial t}}
\newcommand{\xdif}[1] {\frac{\partial #1}{\partial x}}
\newcommand{\ydif}[1] {\frac{\partial #1}{\partial y}}
\newcommand{\zdif}[1] {\frac{\partial #1}{\partial z}}
\newcommand{\xxdif}[1] {\frac{\partial^2 #1}{\partial x^2}}
\newcommand{\yydif}[1] {\frac{\partial^2 #1}{\partial y^2}}

\newcommand{\dt} {\Delta t}
\newcommand{\dtt} {\mbox{$\frac{\Delta t}{2}$}}
\newcommand{\dx} {\mbox{$\Delta x$}}
\newcommand{\dy} {\mbox{$\Delta y$}}

\newcommand{\olds} {\mbox{$\scriptstyle (0)$}}
\newcommand{\news} {\mbox{$\scriptstyle (1)$}}
\newcommand{\meds} {\mbox{$\scriptscriptstyle (\frac{1}{2})$}}
\newcommand{\half} {\mbox{$\scriptstyle \frac{1}{2}$}}

\newcommand{\uold} {\mbox{$U^{\olds}$}}
\newcommand{\vold} {\mbox{$V^{\olds}$}}
%\newcommand{\zold} {\mbox{$\zeta^{\olds}$}}
\newcommand{\unew} {\mbox{$U^{\news}$}}
\newcommand{\vnew} {\mbox{$V^{\news}$}}
%\newcommand{\znew} {\mbox{$\zeta^{\news}$}}

\newcommand{\uuold} {\mbox{$u^{\olds}$}}
\newcommand{\vvold} {\mbox{$v^{\olds}$}}
\newcommand{\uunew} {\mbox{$u^{\news}$}}
\newcommand{\vvnew} {\mbox{$v^{\news}$}}

\newcommand{\uzl} {\mbox{$\zdif{U_{l}}$}}
\newcommand{\vzl} {\mbox{$\zdif{V_{l}}$}}

\newcommand{\uoldp}[1] {\mbox{$U_{{#1}}^{\olds}$}}
\newcommand{\voldp}[1] {\mbox{$V_{{#1}}^{\olds}$}}
\newcommand{\zoldp}[1] {\mbox{$\zeta_{{#1}}^{\olds}$}}
\newcommand{\unewp}[1] {\mbox{$U_{{#1}}^{\news}$}}
\newcommand{\vnewp}[1] {\mbox{$V_{{#1}}^{\news}$}}
\newcommand{\znewp}[1] {\mbox{$\zeta_{{#1}}^{\news}$}}
\newcommand{\zmedp}[1] {\mbox{$\zeta_{{#1}}^{\meds}$}}

\newcommand{\resr} {{\cal R}}
\newcommand{\rhon} {\rho_0}
\newcommand{\drho} {\frac{1}{\rhon}}
\newcommand{\fracs}[2] {\mbox{$\frac{#1}{#2}$}}
%\newcommand{\drho} {\mbox{$\scriptstyle \frac{1}{\rho_{0}}$}}
\newcommand{\deltat} {\mbox{$\tilde{\delta}$}}
\newcommand{\gammat} {\mbox{$\tilde{\gamma}$}}

\newcommand{\AO} {A_{\Omega}}
\newcommand{\dO} {d \Omega}

\newcommand{\beq} {\begin{equation}}
\newcommand{\eeq} {\end{equation}}
\newcommand{\beqa} {\begin{eqnarray}}
\newcommand{\eeqa} {\end{eqnarray}}

\newcommand{\uv} {{\bf U}}
\newcommand{\uvold} {{\bf U^{(0)}}}
\newcommand{\uvnew} {{\bf U^{(1)}}}

\newcommand{\Uold} {U^{(0)}}
\newcommand{\Unew} {U^{(1)}}
\newcommand{\Vold} {V^{(0)}}
\newcommand{\Vnew} {V^{(1)}}
\newcommand{\zold} {\zeta^{(0)}}
\newcommand{\znew} {\zeta^{(1)}}

\newcommand{\Xb} {\hat{X}}
\newcommand{\Yb} {\hat{Y}}
\newcommand{\aush} {A_H}

\newcommand{\az} {\alpha_{z}}
\newcommand{\am} {\alpha_{m}}
\newcommand{\ar} {\alpha_{r}}
\newcommand{\azt} {\tilde{\az}}
\newcommand{\amt} {\tilde{\am}}
\newcommand{\art} {\tilde{\ar}}

\newcommand{\duv} {\Delta {\bf U}}
\newcommand{\dzeta} {\Delta \zeta}
\newcommand{\iv} {{\bf I}}
\newcommand{\ivh} {\hat{\bf I}}
\newcommand{\fv} {{\bf F}}
\newcommand{\uvh} {\hat{\bf U}}
\newcommand{\ffxx} {\tilde{f_x}}
\newcommand{\ffyy} {\tilde{f_y}}

\newcommand{\dvol} {\Delta v}
\newcommand{\cbar} {\bar c}

\newcommand{\tv} {\tau}

%%%%%%%%%%%%%%%%%%%%%%%%%%%%%%%%%%%%%%%%%%%%%%%%%%%%%%%%%%
%%%%%%% hyphenation %%%%%%%%%%%%%%%%%%%%%%%%%%%%%%%%%%%%%%
%%%%%%%%%%%%%%%%%%%%%%%%%%%%%%%%%%%%%%%%%%%%%%%%%%%%%%%%%%

%%%%%%%%%%%%%%%%%%%%%%%%%%%%%%%%%%%%%%%%%%%%%%%%%%%%%%%%%%
%%%%%%% title & author %%%%%%%%%%%%%%%%%%%%%%%%%%%%%%%%%%%
%%%%%%%%%%%%%%%%%%%%%%%%%%%%%%%%%%%%%%%%%%%%%%%%%%%%%%%%%%

\title{Discretization of 2D FEM model}

\author{Georg Umgiesser}

%\date{\today}
%\date{}

%%%%%%%%%%%%%%%%%%%%%%%%%%%%%%%%%%%%%%%%%%%%%%%%%%%%%%%%%%
%%%%%%% document %%%%%%%%%%%%%%%%%%%%%%%%%%%%%%%%%%%%%%%%%
%%%%%%%%%%%%%%%%%%%%%%%%%%%%%%%%%%%%%%%%%%%%%%%%%%%%%%%%%%

\begin{document}

%\baselineskip 20 pt
\maketitle

%\begin{abstract}
%The main features of a newly conceived primitive equation circulation
%model are outlined. The model uses finite elements for spatial
%integration and a semi-implicit algorithm for integration in time.
%The terms treated implicitly are the water levels, the Coriolis term
%and the vertical eddy diffusion. The model uses a combination of
%linear form functions for the expansion of the
%water levels and constant form functions for the velocities.
%Through this combination of form functions the resulting grid resembles
%a staggered finite difference Arakawa B-grid. This
%staggered finite element grid is superior, as far as conservation
%and propagation properties are concerned, to a standard Galerkin method.
%The model has already been used successfully with
%the Venice Lagoon and
%in its full three dimensional formulation will be also
%applied to the Adriatic Sea.
%\end{abstract}


\section*{The Equations}


We start with the momentum equations and the continuity equation:
\beq
\tdif{U} + gH \xdif{\zeta} + RU + X = 0
\eeq
\beq
\tdif{V} + gH \ydif{\zeta} + RV + Y = 0
\eeq
\beq
\tdif{\zeta} + \xdif{U} + \ydif{V} = 0
\eeq
where $x,y,t$ are the space coordinates and time, $U,V$ the
transports in $x,y$ direction, $\zeta$ the water level,
$R$ the friction parameter, $X,Y$ extra terms that cna be treated
explicitly in the following discretization, $g$ the gravitational
acceleration and $H$ the total water depth.

The transports $U,V$ can be obtained from the velocities by
\beq
U = \int u dz \hspace{1cm} V = \int v dz
\eeq
where $u,v$ are the current velocities.

The terms contained in $X,Y$ are the non-linear advective terms, the
Coriolis terms, the wind stress and the lateral eddy friction.
They may be written as
\beq
X = U \xdif{u} + V \ydif{u} - f V - \frac{\tau^x}{\rhon}
	- \aush ( \xxdif{U} + \yydif{U} )
\eeq
\beq
Y = U \xdif{v} + V \ydif{v} + f U - \frac{\tau^y}{\rhon}
	- \aush ( \xxdif{V} + \yydif{V} )
\eeq
where $f$ is the Coriolis parameter, $\tau^x,\tau^y$ the
wind stress, $\rhon$ the reference density of water and
$\aush$ the horizontal eddy viscosity.


\section*{Discretization of the Momentum Equation}


We now chose weighting parameters for the discretization. These parameters
are $a_z$ for the transports in the continuity equation, $a_m$
for the pressure term in the momentum equations and $a_r$ for
the friction term. Associated to these parameters are the parameters
$\azt, \amt, \art$ that are defined as
\beq
\azt = 1 - a_z \quad \amt = 1 - a_m \quad \art = 1 - a_r.
\eeq
All above parameters can take the values from 0 to 1 where
0 means an explicit treatment and 1 a complete implicit treatment.

Discretizing the $x$ momentum equation one obtains
\beq
\frac{\Unew-\Uold}{\dt} + gH [ \am \xdif{\znew} + \amt \xdif{\zold} ]
	+ R [ \ar \Unew + \art \Uold ] + X = 0
\eeq
where the total depth $H$ and the extra terms $X$ are always taken
at the old time step.

Solving for $\Unew$ and introducing the new parameter 
\beq
\delta = \frac{1}{1 + \dt R \ar}
\eeq
we obtain 
\beq
\Unew = \delta (1 - \dt R \art) U 
		- \dt \delta gH [ \am \xdif{\znew} + \amt \xdif{\zold} ]
		- \dt \delta X.
\eeq

Introducing two more auxiliary parameters
\beq
	\gamma = \delta [ 1 - \dt R \art ] \quad
	\beta = \dt \delta gH
\eeq
we finally have for both momentum equations
\beq
\Unew = \gamma U 
		- \beta \am \xdif{\znew} - \beta \amt \xdif{\zold} 
		- \dt \delta X
\eeq
\beq
\Vnew = \gamma V 
		- \beta \am \ydif{\znew} - \beta \amt \ydif{\zold} 
		- \dt \delta Y
\eeq
where the equation in $y$ direction
has been obtained in a similar way obtained in a similar way
as the one in $x$ direction.


\section*{Spatial Integration of the Momentum Equation}


We integrate the momentum equations over one element. For this
we multiply every term with the constant weighting function
$\Psi$ and integrate. Remember that $U,V$ are constant over an
element, and $\zeta$ is varying linearly. The single terms
give
\beq
\int \Psi \Unew \dO = \AO \Unew
\eeq
\beq
\int \Psi X \dO = \int X \dO
\eeq
\beq
\int \xdif{\znew} \dO = \int b_M \znew_M \dO = \AO b_M \znew_M
\eeq
\beq
\int \ydif{\znew} \dO = \int c_M \znew_M \dO = \AO c_M \znew_M
\eeq
where $\Omega$ is the integration domain, $\AO$ the area of the
triangle and $b_M, c_M$ are the constant derivatives of the 
linear form functions $\Phi$
\beq
b_M = \xdif{\Phi} \quad c_M = \ydif{\Phi}.
\eeq

We therefore obtain
\beq
\Unew = \gamma U 
		- \beta \am b_M \znew_M - \beta \amt b_M \zold_M 
		- \dt \delta \Xb
\eeq
\beq
\Vnew = \gamma V 
		- \beta \am c_M \znew_M - \beta \amt c_M \zold_M 
		- \dt \delta \Yb
\eeq
with
\beq
\Xb = \frac{1}{\AO} \int X \dO \quad
\Yb = \frac{1}{\AO} \int Y \dO.
\eeq


\section*{Integration of the Continuity Equation}

We first integrate the continuity equation over one element
and obtain
\beq
\int \Phi \xdif{\znew} \dO = \int b_M \znew_M \dO = \AO b_M \znew_M
\eeq





\end{document}
