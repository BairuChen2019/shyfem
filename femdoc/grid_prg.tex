
The GRID program allows not only to visualize 
the |GRD| files but also provide a graphical user interface 
to manipulate the different items of the grid (Nodes, Elements, Lines).
The command line and the options available for this program are following reported.

\begin{verbatim}
grid [-options] [files] 

Options :
  -o   do not outline elements       -f   fill elements with color  
  -k   do extra checking             -u   check if nodes are used   
  -T   show type instead of depth    -c#  use color descrple #         
  -h   print this help screen        -a   ask for file names        
  -d   display  (only X11)           -g   geometry (only X11)       
  -M#  scale color to depth #        -S#  size of color descrple is #  
  -N#  scale factor for nodes is #   -V#  scale factor for vectors is #
  -C   color nodes and lines     
\end{verbatim}

\textbf{General GUI commands}\\
	scroll 		zoom in and out \\
	right click 	select item\\
	left click	confirm item\\
	up arrow	increase the node size\\
	down arrow 	decrease the node size\\

The main GUI menu is compsed by\\
\descrp{|File|}\\
\descrp{|View|}\\
\descrp{|Show|}\\
\descrp{|Node|}\\
\descrp{|Element|}\\
\descrp{|Line|}\\
\descrp{|Change|}\\

\textbf{File menu}\\
\descrp{|Cancel|}
\descrptext{%
Obsolete command
}
\par
\descrp{|Refresh|}
\descrptext{%
Refresh the screen view (to be done to view the last change to the grid)
}
\par
\descrp{|Print|}
\descrptext{%
Create a Black and White PostScript of the grid |plot.ps|
}
\par
\descrp{|Save|}
\descrptext{%
Save changes in |save.grd|
}
\par
\descrp{|Exit|}
\descrptext{%
Quit GRID program
}
\par

\textbf{View menu}\\
\descrp{|Zoom Window|}
\descrptext{%
Zoom in a delimited window defined by left clicking two points (left-bottom and righ-top)  
}
\par
\descrp{|Zoom in|}
\descrptext{%
Obsolete command replaced by mouse scroll
}
\par
\descrp{|Zoom out|}\descrptext{%
Obsolete command replaced by mouse scroll
}
\par
\descrp{|Total View|}
\descrptext{%
Go back to the total view of the grid
}
\par
\descrp{|Move|}
\descrptext{%
Obsolete command
}
\par

\textbf{Show menu}\\
\descrp{|Show Node|}
\descrptext{%
All the items selected by rigth clicking will be nodes
}
\par
\descrp{|Show Element|}
\descrptext{%
All the items selected by rigth clicking will be elements
}
\par
\descrp{|Show Line|}
\descrptext{%
All the items selected by rigth clicking will be lines
}
\par

\textbf{Node menu}\\
\descrp{|Make Node|}
\descrptext{%
Create a new node by left clicking
}
\par
\descrp{|Del Node|}
\descrptext{%
Delete a node by selecting it (right click) and confirming it (left click). |Refresh| to see the changes.
}
\par
\descrp{|Move Node|}
\descrptext{%
Move a node in a new position. 
Select it (right click) and confirm it (left click), give the new position (left click). |Refresh| to see the changes.
}
\par
\descrp{|Unify Node|}
\descrptext{%
Unify two different nodes. 
Select the first node you want to unify (right click) and confirm it (left click), select the second node (left click). |Refresh| to see the changes.
}
\par

\textbf{Element menu}\\
\descrp{|Make Element|}
Create a new element.
Create new nodes (left click) or select and confirm (right-left click) each node of the new element, clicking twice on the last one to close the element. The element has to be created in anti-clockwise sense.
\descrptext{%

}
\par
\descrp{|Del Element|}
Remove the element but not its nodes.
\descrptext{%

}
\par
\descrp{|Remove Element|}
Remove the element and its nodes.
\descrptext{%

}
\par

\textbf{Line menu}\\
\descrp{|Make Line|}
\descrptext{%
Create a new line.
Create new nodes (left click) or select and confirm (right-left click) each node of the new line, clicking twice on the last one. In case of a close line it has to be created in anti-clockwise sense.
}
\par
\descrp{|Del Line|}
\descrptext{%
Remove the line but not its nodes.
}
\par
\descrp{|Remove Line|}
\descrptext{%
Remove the line and its nodes.
}
\par
\descrp{|Split Line|}
\descrptext{%
Split the line in two parts.
}
\par
\descrp{|Join Line|}
\descrptext{%
Join two lines in one.
}
\par
\descrp{|Del Node|}
\descrptext{%
Delete the node from line but not from the domain.
}
\par
\descrp{|Remove Node|}
\descrptext{%
Remove the node from line.
}
\par
\descrp{|Insert Node|}
\descrptext{%
Insert a new node on the line.
}
\par

\textbf{Change menu}\\
\descrp{|Change Depth|}
\descrptext{%
Change the depth of the selected item.
}
\par
\descrp{|Change Type|}
\descrptext{%
Change the type of the selected item.
}
\par






