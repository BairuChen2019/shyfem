
\newcommand{\version}{6_1_47}
\newcommand{\shydir}{shyfem-\version}

The source code of the model is provided in a file named 
|\shydir.tar.gz| or similar, depending on the version
of the code. In this case the version is \version.
The file can be downloaded from the SHYFEM 
web-site\footnote{http://www.ismar.cnr.it/shyfem/}, or
by contacting directly its authors.

The source code is composed mainly of Fortran 77 files, but
files written in C, Fortran 90, Perl and Shell scripts
are also present.

In order to use the model you have to compile it in a
Linux Operating System. Several software products must be present
in order to be able to compile the model. Please refer to
the documentation of your Linux distribution for installing these programs.

Please see the next section for software that is
helpful for running and using the model.
However, the software absolutely necessary is the following:

\begin{itemize}
\item A Fortran 77 and 90 compiler. Supported compilers are
the Gnu |g77| and |g90| compilers, the new Gnu compiler |gfortran|,
the Intel Fortran compiler |ifort| or the Portland group Fortran compiler.
\item The package |make| is required for compilation.
\item The |perl| interpreter, the |bash| shell and the |gcc| c compiler
are necessary for compiling.
\end{itemize}

Please note that you might already have everything available in your
Linux distribution, with the exception of the fortran compiler.
Please see the section on ``Needed software'' for more information
on the recommended software you should have installed on your computer.

Once you have downloaded the model open a shell (console) as 
a normal user, go to the directory where you have downloaded the
shyfem distribution and run the following command:

\begin{citation}
\item |tar -xzvf \shydir.tar.gz|
\end{citation}

At this point a new folder named |\shydir|
has been created. Move into this directory (|cd \shydir|) 
and open the file
|Rules.make| with a text editor.
In this file all the compilation options can be set.
These are the sections inside it:

\begin{itemize}
\item |Parameters|. In this section you have to set the
maximum number of nodes (|NKNDIM|) and elements (|NELDIM|)
used by your grids. You might also have to set also the maximum number 
of elements attached to
a node (|NGRDIM|) and the maximum bandwidth (|MBWDIM|) of the
z-level matrix. You can find these numbers when you create the basin file
with |vpgrd|. Finally you have to specify
the maximum number of vertical levels (|NLVDIM|).
It is advisable to set this value close to the desired number of
vertical levels, since it affects the model speed performance. So, if
you want to run the model in 2D mode, please set |NLVDIM| to 1.
\item |Compiler|. Set the compiler you want to use.
\item |Parallel compilation|. Some parts of the code are parallelized
with OpenMP statements. Here you can set if you want to use it or not.
Intel and Gnu gfortran compilers accept OpenMP statements.
\item |Solver for matrix solution|. There are three different solvers
implemented. 
The |GAUSS| solver is the most robust and best tested solver,
but it is quite slow. The |PARDISO| solver needs an
external library available at the Intel 
web-site\footnote{http://software.intel.com/en-us/articles/code-downloads/},
that is freely downloadable for non-commercial use. 
The Pardiso solver is parallelized,
but it seems to be a little slower than |SPARSKIT| solver.
The |SPARSKIT| solver is the suggested solver, since it seems to be the 
fastest one. However, if you are ever in doubt about your results you might
to revert back to the |GAUSS| solver and check the results.
\item |NetCDF library|. If you want the output files in NetCDF format
you need the NetCDF library.
\item |GOTM library|. The GOTM turbulence model is already included
in the code. However, a newer and better tested version is available as
an external module. In order to use it please set this variable
to true. This is the recommended choice.
\item |Ecological module|. This option allows for the inclusion
of an ecological module into the code. Choices are between
|EUTRO|, |ERSEM| and |AQUABC|. Please refer to information given 
somewhere else for how to run these programs.
\item |Compiler options|. Here several sections are present, 
one for each supported compiler. Normally it should not be necessary to 
change anything beyond this point.
\end{itemize}

Once you have set all these options you can run the command

\begin{citation}
\verb|make fem|
\end{citation}

to compile everything.

Another useful command is |make clean| to delete files of previous compilations.

It is advisable to make in your home directory a symbolic link named
|shyfem| to the directory, where you have installed SHYFEM. Otherwise
some of the commands and shell scripts might not work properly. This
can be done with the command

\begin{citation}
\verb|ln -s \shydir shyfem|
\end{citation}

from your home directory. If there is already such a link existing you first
have to delete it (|rm shyfem|).

Another useful option is to add the fembin path to your default paths
to have the main utility commands always available. To do this
open your |.bashrc| file in the home directory
and add the following lines at the end of the file:

\begin{verbatim}
FEMDIR=$HOME/fem
PATH=$PATH:$FEMDIR/fembin
export FEMDIR PATH
\end{verbatim}

If you have not set a symbolic link from |fem| to the actual model directory,
please substitute |fem| with the appropriate directory |fem_VERS_N_nn|.

Summarizing, the following steps have to be taken in order to
properly install the model on your computer:
\begin{itemize}
\item download the model and unpack it somewhere
\item |cd| to the model directory and adjust |Rules.make|
\item compile the model with |make fem|
\end{itemize}

In addition, you also might
\begin{itemize}
\item set a symbolic link from |fem| to the model directory
		|ln -s fem_VERS_N_nn fem|
\item adjust the file |.bashrc| adding the |FEMDIR| shell variable 
	and a new entry for the |PATH|
\end{itemize}
