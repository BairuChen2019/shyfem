
%------------------------------------------------------------------------
%
%    Copyright (C) 1985-2018  Georg Umgiesser
%
%    This file is part of SHYFEM.
%
%    SHYFEM is free software: you can redistribute it and/or modify
%    it under the terms of the GNU General Public License as published by
%    the Free Software Foundation, either version 3 of the License, or
%    (at your option) any later version.
%
%    SHYFEM is distributed in the hope that it will be useful,
%    but WITHOUT ANY WARRANTY; without even the implied warranty of
%    MERCHANTABILITY or FITNESS FOR A PARTICULAR PURPOSE. See the
%    GNU General Public License for more details.
%
%    You should have received a copy of the GNU General Public License
%    along with SHYFEM. Please see the file COPYING in the main directory.
%    If not, see <http://www.gnu.org/licenses/>.
%
%    Contributions to this file can be found below in the revision log.
%
%------------------------------------------------------------------------

\documentclass{article}

\usepackage{shortvrb}

\newcommand{\psp}{{\tt pspost}}
\newcommand{\VERSION}{1.2}

\MakeShortVerb{\|}

%%%%%%%%%%%%%%%%%%%%%%%%%%%%%%%%%%%%%%%%%%%%%%%%%%%%%%%%% front matter

\title{%
	\psp
	\\Doing Graphics in PostScript
	\\with Fortran and C
	}

\author{%
	Georg Umgiesser
	\\ISDGM-CNR, S. Polo 1364
	\\30125 Venezia, Italy
	\vspace{1cm}
	\\Version \VERSION
	}

%\address{ISDGM-CNR}

%%%%%%%%%%%%%%%%%%%%%%%%%%%%%%%%%%%%%%%%%%%%%%%%%%%%%%%%% document

\begin{document}

\pagenumbering{roman}
\pagestyle{plain}

\maketitle

\begin{abstract}
A simple graphic plotting package is presented that can be used for
the creation of PostScript graphics. The routines are callable from
Fortran and C.

Only the basic plotting commands have been implemented. The library
allows you to plot lines and points, fill arbitrary shapes
with arbitrary color and write text with an arbitrary point size.
It also allows for producing more pages in one plot. Color can be
used as gray scale or through the RGB and HSB color spaces.
The coordinate system may be set to best adjust to the drawing.
Clipping graphics in a given rectangle is implemented.
\end{abstract}

\thispagestyle{empty}

\newpage

\tableofcontents

\newpage

\section*{Disclaimer}
\addcontentsline{toc}{section}{Disclaimer}


\begin{quotation}
  									 
   Copyright (c) 1992-1997 by Georg Umgiesser				 
  									 
   Permission to use, copy, modify, and distribute this software	 
   and its documentation for any purpose and without fee is hereby	 
   granted, provided that the above copyright notice appear in all	 
   copies and that both that copyright notice and this permission	 
   notice appear in supporting documentation.				 
  									 
   This file is provided AS IS with no warranties of any kind.		 
   The author shall have no liability with respect to the		 
   infringement of copyrights, trade secrets or any patents by		 
   this file or any part thereof.  In no event will the author		 
   be liable for any lost revenue or profits or other special,		 
   indirect and consequential damages.					 
  									 
   Comments and additions should be sent to the author:			 
  									 
	\begin{verbatim}
  			Georg Umgiesser					 
  			ISDGM/CNR					 
  			S. Polo 1364					 
  			30125 Venezia					 
  			Italy						 
  									 
  			Tel.   : ++39-41-5216875			 
  			Fax    : ++39-41-2602340			 
  			E-Mail : georg@lagoon.isdgm.ve.cnr.it		 
	\end{verbatim}
\end{quotation}

\section*{Availability}
\addcontentsline{toc}{section}{Availability}


The library \psp{} is available free of charge bye anonymous ftp.
Connect to
|ftp.isdgm.ve.cnr.it| and look in the directory |/pub/|\psp{}.
Please read the README file of the distribution. Only source code
is available.

The library should compile out of the box for nearly all Unix-like
systems. Please follow the instructions in the README file.

Please send bug reports to the author (|georg@lagoon.isdgm.ve.cnr.it|).


\newpage

\pagenumbering{arabic}

