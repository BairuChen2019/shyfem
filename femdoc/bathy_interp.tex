

After the grid creation, with mesh or other programs, the interpolation
of bathymetry file is necessary.
To interpolate bathymetry, a |GRD| file with single points containig
depth values has to be available. This file, together with the basin
onto which the bathymetry has to be interpolated, has to be specified
for the program basbathy. The simplest call is: \\

        |baselab mesh2 bathy|\\

where |bathy.grd| is the |GRD| file with the bathymetry values and
|mesh2| is the basin for which to interpolate the bathymetry.
Different types of interpolation can be used. Please run
|baselab -h| for more options.

The |GRD| output file will be in "new.grd".\\


%\subsection{Create basin for FEM model (bandwidth optimization)}

%Before proceeding to the simulations we must first create a
%representation of the basin suitable for the finite element model.
%
%In order to create the finite element reppresentation of the
%grid, please run "vpgrd mesh3". This creates a file mesh3.bas.
%This is a binary file suitable for being read by the finite
%element model.\\

%        vpgrd mesh3\\

