
The SHYFEM program is designed to work with most of the compilers
that are available. Normally there should be no problems with
compatibility. However you have to keep in mind some points that are
listed below.

The supported compilers for SHYFEM are

\begin{itemize}

\item {\bf g77} This the old GNU compiler. It is Fortran 77 only. It
is also a little outdated and is not supported anymore by the major
distributions. If you do not need software that relies on Fortran 90 it
is still possible to use it. However, GOTM and some ecological modules
are relying on Fortran 90 and cannot be used with the g77 compiler.

\item {\bf gfortran} This is the actual GNU compiler. It also
supports Fortran 90. This is the compiler that you will find in recent
distributions.

\item {\bf ifort} This is the INTEL compiler. You will have to download
and install it on your own. It is very efficient and normally faster
than the GNU compiler.

\item {\bf pgf90} This is the Portland group compiler. It is a commercial
compiler. It creates very efficient code.

\end{itemize}

Whatever compiler you choose to use, you have to set your choice in
the |Rules.make| file. Even if not desirable, different compilers can
give you slightly different results in the computations. This is due to
the different optimizations enabled and maybe a different treatment of
accuracy and round off. Other compatibility issues are the following.

\begin{itemize}

\item With |g77| and |ifort| it is possible to open the same file in
read only mode more than once. This is useful, e.g., if you have two
open boundaries, but you want to prescribe the same value on these
two boundaries. With |gfortran| or |pgf90| you cannot do this. A file,
even in read only mode, can be opened only once. In the above example
you therefore have to copy the input file to a new name (duplicate it)
and then prescribe the two different files as boundary conditions.

\item With |gfortran| it is very difficult to decide if a file is
formatted or unformatted. Some modules allow the use of either formatted
or unformatted input files, where the check on the file type is made
via software. In case of |gfortran| this may not work reliably. The only
solution to this problem is to specify the file type directly in the code.

\item Objects generated during compilation and libraries used in linking
are normally not compatible between compilers. What this means is that,
when you switch compiler, you will have to recompile everything with
|make cleanall; make fem|. Otherwise you will encounter errors during
the linkage process.

\item Unformatted files are normally not portable between different
compilers. You normally cannot use a basin file created with programs
compiled with one compiler together with a program compiled with another
compiler. The same is true for unformatted data files (initial conditions,
wind and meteo forcing, etc.).

If you have problems reading a basin file, try |basinf|. If this is
not working chances are high that you have the problem described above.
In case of unformatted data files the diagnosis is not so easy. In any
case, you can solve the problem recompiling all programs with the commands
|make cleantotal; make fem| and then re-creating all unformatted files
with the newly compiled programs. In case of the basin file, you will
have to run the pre-processor on the grid again.

If you have obtained unformatted data files from others, then there is
really no easy solution to this problem. Exchanging unformatted files
between different computers and compilers is never a good idea.

\item A similar problem exists if you switch files between different
architectures (32 bit and 64 bits), even if created with the same
compiler. These files are normally not portable.

\item Nan values (Not a Number) are treated differently between different
compilers. Nan values are created if a not well defined operation is
executed (divide by 0 or square root of a negative number). All compilers
above (except |pgf90|) treat Nans to be not comparable to any number.
This means that a logical expression |a.eq.a| is always false if |a|
is a Nan. However the |pgf90| compiler treats Nans to be comparable
to any other number. So, an expression like |a.ne.a| will evaluate to
true. SHYFEM includes code to handle these problems gracefully, but
incompatibilities might still show up.

\item In parallel execution you might get a segmentation fault during
execution. This is normally due to limited stack size. You can change
the behavior by increasing the stack size (|ulimit -s unlimited|)
on the console before running the program. Compilers may behave
differently. Please see also the section on parallel execution in the
file |Rules.make|.

\end{itemize}






