
In order to compile the model you will first have to adjust some settings
in the |Rules.make| file. Assuming that you are already in the SHYFEM
root directory (in our case it would be \ttt{\shydir}), open the file
|Rules.make| with a text editor.  In this file the following options
can be set:

\begin{itemize}

\item |Parameters|. In this section you have to set the maximum number
of nodes (|NKNDIM|) and elements (|NELDIM|) used by your grids. You
might also have to set also the maximum number of elements attached to
a node (|NGRDIM|) and the maximum bandwidth (|MBWDIM|) of the z-level
matrix. You can find these numbers when you create the basin file with
|vpgrd|. Finally you have to specify the maximum number of vertical levels
(|NLVDIM|).  It is advisable to set this value close to the desired number
of vertical levels, since it affects the model speed performance. So,
if you want to run the model in 2D mode, please set |NLVDIM| to 1.
For all other possible parameter settings please have a look at |param.h|.
However, you should never directly change this file. Always make changes
to the parameters in the |Rules.make| file.

\item |Compiler|. Set the compiler you want to use. Please see also
the section on needed software and the one on compatibility problems to
learn more about this choice.

\item |Parallel compilation|. Some parts of the code are parallelized
with OpenMP statements. Here you can set if you want to use it or not.
All supported compilers (except |g77|) accept OpenMP statements.

\item |Solver for matrix solution|. There are three
different solvers implemented.  The |GAUSS| solver is the
most robust and best tested solver, but it is quite slow. The
|PARDISO| solver needs an external library available at the Intel
web-site\footnote{http://software.intel.com/en-us/articles/code-downloads/},
that can be freely downloaded for non-commercial use.  The Pardiso
solver is parallelized, but it seems to be a little slower than the
|SPARSKIT| solver.  The |SPARSKIT| solver is the recommended solver,
since it seems to be the fastest one. However, if you are ever in doubt
about your results you might want to revert back to the |GAUSS| solver
and check the results.

\item |NetCDF library|. If you want output files in NetCDF format
you need the NetCDF library.

\item |GOTM library|. The GOTM turbulence model is already included in
the code. However, a newer and better tested version is available as an
external module. In order to use it please set this variable to true. This
is the recommended choice. You will need a Fortran 90 compiler to enable
this choice.

\item |Ecological module|. This option allows for the inclusion of an
ecological module into the code. Choices are between |EUTRO|, |ERSEM|
and |AQUABC|. Please refer to information given somewhere else on how
to run these programs.

\item |Compiler options|. Here several sections are present, one for
each supported compiler. Normally it should not be necessary to change
anything beyond this point.

\end{itemize}

Once you have set all these options you can start compilation with

\begin{code}
    make clean
    make fem
\end{code}

This should compile everything. In case of a compilation error you will
find some messages during compilation and also at the bottom of the output,
where a check is run to see, if the main routines have been compiled.

Please remember that you will always have to run the commands above
when you change settings in the |Rules.make| file. If you only change
something in the code, or if you only change dimension parameters, it
might be enough to run only |make fem|, which only compiles the necessary
files. However, if you are in doubt, it is always a good idea to run
|make clean| or |make cleanall| before compiling, in order to start from
a clean state.

