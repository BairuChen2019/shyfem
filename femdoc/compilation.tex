
In order to compile the model you will first have to adjust some settings
in the |Rules.make| file. Assuming that you are already in the SHYFEM
root directory (in our case it would be \ttt{\shydir}), open the file
|Rules.make| with a text editor.  In this file the following options
can be set:

\begin{itemize}

\item |Compiler profile|. Set the level of verbosity of the messages. Use
|SPEED| if you want the maximum performances. Use the other options, in
case of errors, to have more informations.

\item |Compiler|. Set the compiler you want to use. Please see also
the section on needed software and the one on compatibility problems to
learn more about this choice. It is advaisable to use the same type of
compiler for C and Fortran.

\item |Parallel compilation|. Some parts of the code are parallelized
with OpenMP statements. Here you can set if you want to use it or not.

\item |Solver for matrix solution|. There are three
different solvers implemented. |SPARSKIT| is an iterative solver, 
quite fast, and is the default option.
The |GAUSS| solver is a robust direct solver, but it is quite slow. 
|PARDISO| is set as direct solver but can be used as iterative
solver as well. It can be fast, but it is not included in the code,
since it is not provided with a compatible licence. In order to
use it, you need an external library (dynamically linked) provided 
with the Intel MKL.

\item |NetCDF library|. If you want output files in NetCDF format
you need the NetCDF library.

\item |GOTM library|. The GOTM turbulence model is already included in
the code. However, a newer and better tested version is available as an
external module. In order to use it please let this variable to true. This
is the recommended choice. You will need a Fortran 90 compiler to enable
this choice.

\item |Ecological module|. This option allows for the inclusion of an
ecological module into the code. Choices are between |EUTRO|, |ERSEM|
and |AQUABC|. Please refer to information given somewhere else on how
to run these programs.

\item |Fluid mud|. This is an experimental feature. Don't use it
if you are not a developer.

\end{itemize}

Once you have set all these options you can start compilation with

\begin{code}
    make clean
    make fem
\end{code}

This should compile everything. In case of a compilation error you will
find some messages during compilation and also at the bottom of the output,
where a check is run to see, if the main routines have been compiled.

Please remember that you will always have to run the commands above
when you change settings in the |Rules.make| file. If you only change
something in the code, or if you only change dimension parameters, it
might be enough to run only |make fem|, which only compiles the necessary
files. However, if you are in doubt, it is always a good idea to run
|make clean| or |make cleanall| before compiling, in order to start from
a clean state.

