
Before you can start using the model you have to create a numerical grid.
This step is more difficult for models that work on unstructured grids
(like finite element models) than for finite difference models, where
often it is enough to have a regular gridded bathymetry to start running
simulations.

This chapter describes the steps needed to create a numerical grid for SHYFEM.
The GRID program is a graphic utility for the visualization and the manipulation of the GRD files.

The grid files used by SHYFEM are two:\\

|filename.grd| formatted\\
|filename.bas| unformatted\\

Using the GRID program, the nodes, lines and elements can be modified by hand, 
acting on the |GRD| file. The final aim is to create a (unformatted) basin 
file |BAS| directly readable by SHYFEM.

The |GRD| file can be composed of 3 different structures: Nodes, Elements, Lines.

- Node section, containing the nodes information and coordinates\\
- Element section, containing the elements information and the their nodes\\
- Line section, containing the domain contour infomation and nodes\\

The presence of these structures depends on type of GRD file, for example,
in a boundary line the structures will be the line and its nodes.
For more details on the format, please, refer to GRD file appendix B.1.\\

In the following sections the steps that have to be carried out 
to create a |GRD| file and a |BAS| file are described

\begin{enumerate}

\item obtain raw digital data of the coastline and the bathymetry and
  convert them into a |GRD| format readabl by the provided routines

\item smooth and reduce the coastline if needed

\item create the numerical grid with the automatic mesh generator

\item regularize the grid

\item interpolate bathymetry onto the created grid

\item create the unformatted basin file

\end{enumerate}

Some of these steps can be skipped if you already
have a finite element grid. If you already have a grid with bathymetry
information you can jump to point 6. However, you will have to convert
your grid into the |GRD|format used by SHYFEM. 

% UNDER CONSTRUCTION
For some of the most common unstructured grid
formats routines are available to convert between these formats and the
|GRD|format. In any case, the |GRD|format is quite easy to parse and
write, so you might be able to write a transformation routine yourself.
%%%%%%%%%%%%%%%%%%%%%%%

In the following a description is given what you have to do if you start
from scratch. Please refer to the section on other programs to create
a grid for conversion routines.


%old text

%If you are in the happy position
%to already have a numerical finite element grid, then you can jump ahead
%to the section on how to transform to/from other grid formats (UNDER COSTRUCTION). 
%You might still have to interpolate a bathymetry onto the numerical grid, so you
%will have to refer to the section on interpolating bathymetry. In any
%case, at the end you will have to create a (unformatted) basin file,
%so that SHYFEM is able to read in the information.
