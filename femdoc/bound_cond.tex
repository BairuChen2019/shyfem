

In order to have a more meaningfull simulation, we need to specify
boundary conditions. In this section we will deal with the open boundary
conditions, e.g., the conditions at the place where the basin comunicates
with other water bodies (e.g., for lagoons it could be the inlets).

For every boundary condition one section |$bound| must be specified. Since
you can have more than one open boundary you must specify also the number
of your boundary, e.g., |$bound1|, |$bound2| etc. Inside every section
you can then specify the various parameters that characterize your boundary.

Basically there two types of open boundary conditions. Either the water
level or the discharges (fluxes) can be specified. The parameter that
decides the type of boundary is |ibtyp|. A value of one indicates water
levels, instead a value of 2 or 3 indicates fluxes. If you specify
discharges entering at the border of the domain, |ibtyp| = 2 should be
specified. Otherwise, if there are internal sources in the basin then
|ibtyp| = 3 must be used. If you do not define this parameter, a value of 1
will be used and water levels will be specified.

The only compulsary parameter in this section is the list of boundary
nodes.  You do this with the parameter |kbound|. 
In the case of |ibtype| 1 or 2 at least two nodes must be
specified, in order to give an extension of the boundary. The numeration
of the boundary nodes must be consecutive and with the basin on its
left side when going along the boundary nodes.  In the case of |ibtyp|
= 3 even a single point can be given.

The boundary values you want to give are normally specified through 
a a file with a time series. You give the name of the file that contains
the time series with the parameter |boundn|. 
An example with two boundaries can again be found in 
\Fig\figref{example}. Here water levels are prescribed and the values
for the water levels are read from a file |levels1.dat|.

If the values on the boundary
you want to impose can be described through a simple sinus function, you
can also give the bounadry values specifying the parameters for the
sinus function. An example of a water level boundary with a tide of
$\pm 70 cm$ and a period of 12 hours (semi-diurnal) is given in
\Fig\figref{bound}. Note thet |zref| gives the average water level of the
boundary. If you specify |ampli|=0 you get a constant boundary value
of |zref|.

\begin{figure}[ht]
\begin{verbatim}
$bound1
      ibtyp = 1   kbound = 23 25 28
      ampli = 0.70  period = 43200  phase = 10800  zref = 0.
$end
\end{verbatim}
\caption{Example of a boundary with regular sinusoidal water levels.
The pahse of 10800 (3 hours) makes sure that the simulation starts at
slack tide when the basin is completely full.}
\label{fig:bound}
\end{figure}








