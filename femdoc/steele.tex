%\documentclass{article}
%\usepackage{a4}
%\usepackage{shortvrb}
%
%\MakeShortVerb{\|}
%\newcommand{\beq}{\begin{equation}}
%\newcommand{\eeq}{\end{equation}}
%
%\begin{document}

\newcommand{\IL}{\frac{I}{I_o}}
\newcommand{\IiL}{\frac{I_i}{I_o}}
\newcommand{\IiS}{\frac{I_i}{I_s}}
\newcommand{\IaL}{\frac{I_a}{I_o}}
\newcommand{\ekz}{e^{-kz}}
\newcommand{\ekH}{e^{-kH}}
\newcommand{\PO}{\overline{P(t)}}
\newcommand{\POO}{\overline{P}}



\subsection{Light attenuation formula by Steele and Di Toro}

The well known light limitation function proposed by Steele
is given as:

\beq \label{SteeleOrig}
P = \IL e^{1-\IL}
\eeq
where $I$ is the light intensity and $I_o$ the optimal light
intensity. $P$ is the limiting function and takes values
between 0 and 1.

In this form, $P$ is a function of depth $z$ and of time $t$
($P = P(z,t)$) since the light intensity depends on depth and time
($I = I(z,t)$). The depth dependence of $I$ can be written as
\beq \label{DepthExt}
I(z) = I_i e^{-kz}
\eeq
where $I_i$ is the incident light intensity 
on the surface (still dependent on time) and
$k$ is an extinction coefficient. Inserting (\ref{DepthExt})
into (\ref{SteeleOrig}) gives the equation
\beq \label{start}
P(z,t) = \IiL \ekz e^{1-\IiL\ekz} = e \IiL \ekz e^{\IiL\ekz}.
\eeq

We now compute the average of $P$ over the water column.
This gives
\[
\PO = \frac{1}{H} \int_0^H P(z,t) dz = 
	\frac{e \IiL}{H} \int_0^H \ekz e^{-\IiL\ekz} dz.
\]
With the substitution $x = \ekz$ and therefore 
$dx/dz = -k \ekz = -kx$ the integral can be transformed into
\[
\PO = \frac{e \IiL}{H} \int_0^H x e^{-\IiL x} \frac{-1}{kx} dx
	= - \frac{e \IiL}{kH} \int_0^H e^{-\IiL x} dx.
\]
Solving this integral gives finally
\beq \label{Steele}
\PO = \frac{e}{kH} \left[ e^{-\IiL x} \right]_0^H
	= \frac{e}{kH} \left[ e^{-\IiL \ekz} \right]_0^H
	= \frac{e}{kH} \left[ e^{-\IiL \ekH} - e^{-\IiL} \right].
\eeq
This is the depth integrated form of the Steele limiting function
for instantaneous light.

If we want to work with the average light over one day, then
equation (\ref{Steele}) can be easily averaged over one day.
If $T$ is the averaging period (one day), $f$ the fraction of the day with
daylight and $I_i$ is approximated with a step function, 0 at night
and $I_a$ during daytime, then we can write
\beq \label{DiToro}
\POO = \frac{1}{T} \int \PO dt 
	= \frac{1}{T} fT \frac{e}{kH} \left[ e^{-\IaL \ekH} - e^{-\IaL} \right]
	= \frac{e}{kH} f \left[ e^{-\IaL \ekH} - e^{-\IaL} \right].
\eeq
Equation (\ref{DiToro}) represents the limiting function given by Di Toro
and used in the EUTRO program of WASP. Therefore, equation (\ref{DiToro})
is just the Steele limiting function, but using $I_a$, the average 
incident light intensity over daylight hours instead the 
instantaneous incident light intensity $I_i$ used in the original
Steele formula (\ref{Steele}).


\subsection{Light attenuation formula by Smith}

EUTRO uses also a light limitation formula by Smith. In the manual
this is given as 
\beq \label{Smith}
\PO = \frac{e}{kH} \left[ e^{-\IiS \ekH} - e^{-\IiS} \right].
\eeq
where now $I_S$ is the optimal light intensity which is not a constant but is
continuously adjourned by the program. Again, $I_i$ is
the instantaneous light intensity at the surface.

$I_i$ is given as
\beq \label{IncLight}
I_i = \frac{\pi I_t}{2f} \sin (\frac{\pi t}{f})
\eeq
between 0 and $f$ (daylight) and $I_i=0$ otherwise.
Averaging (\ref{IncLight}) over one whole day (0--1) gives
\[
\overline{I_i}^{day} = \int_0^f \frac{\pi I_t}{2f} \sin (\frac{\pi t}{f}) dt
	= I_t
\]
and averaging only over daylight hours gives
\[
\overline{I_i}^{daylight} = 
	\frac{1}{f} \int_0^f \frac{\pi I_t}{2f} \sin (\frac{\pi t}{f}) dt
	= \frac{I_t}{f} = I_a.
\]
This shows that $I_t$ in equation (\ref{IncLight}) is the average
light intensity at the surface over one whole day (|ITOT| in Eutro)
and that $I_a = I_t/f$ is the average light intensity at the surface
over daylight hours. This must be taken into account when the
Di Toro formulation is used.

Note that in EUTRO the actual formula used is 
$I_a = 0.9 I_t/f$ where the parameter 0.9 probably accounts
for some losses during the integration. The corresponding
variables in EUTRO are |FDAY| for $f$ and |IAV| for $I_a$.




%\end{document}
