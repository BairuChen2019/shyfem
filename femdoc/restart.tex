
The model solves the shallow water equations, forwarding in time an
initial state of the hydrodynamical system. This state is composed by all
the model independent variables.  The restart routines allow to load the
initial values for these variables, which, otherwise, would be unknown
and set to zero.  The restart allows also to write the state in a file
one time or at different time intervals.

To load a restart file the parameters |itrst| and |restrt| must be
set. The first must be included in the section |para| and is the time (in
seconds from the initial date or in date label) relative to the restart
record to read.  The second must be specified in the section |name|
and is the name of the restart file, which must have extension |rst|.
For example:

\begin{code}
$para
...
itrst = '2016-01-25::06:00'
...
$end

...

$name
...
restrt = 'myrestart.rst'
...
$end
\end{code}

A new restart file can be created by using |itmrst|, the time to write
the first restart record, and |idtrst|, the time step between different
records. Both the parameters must be specified in the |para| section.
For example:

\begin{code}
$para
...
itmrst = '2016-01-26::06:00'
idtrst = '6h'
...
$end
\end{code}

Finally if you want to check the records contained in a restart file,
you can use |rstinf|:

\begin{code}
    rstinf myrestart.rst
\end{code}


For more information see the description of the parameters in the appendix.

