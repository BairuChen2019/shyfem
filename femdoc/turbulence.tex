
In the Reynolds equations turbulent eddy diffusivities and viscosities are
introduced into the equations that must be parameterized and given some
value. Moreover
SHYFEM assumes the hydrostatic approximation. Therefore, there is the need 
to parameterize the non-hydrostatic effects. These are considered
sub-scale processes which are mainly of convective nature.

Vertical eddy viscosities and diffusivities have to be defined 
if there is the intent to model the turbulence effects.
These vertical eddy viscosities and diffusivities can be set to 
constant values, defining |vistur| and |diftur| in the |$para| section.
There is also the opportunity to compute, at each timestep, 
variable values of them, using the turbulence closure module.

The parameter that has to be set in order too choose the 
turbulence scheme is |iturb| in the |$para| section.. 
If |iturb|=0 the vertical eddy viscosity and eddy diffusivity are 
set constant (default 0) and must be defined in |vistur| and |diftur|.

If |iturb|=1 the turbulence closure scheme applied is the $k-\epsilon$ model. 
If |iturb|=2 the GOTM turbulence closure module is used. In this case 
the file |gotmturb.nml| must be provided that sets all necessary parameters. 
This file must be declared in the section |$name| for the item |gotmpa|.

A default |gotmturb.nml| file is provided and it allows the computation of 
the vertical eddy viscosity and eddy diffusivity by means of the 
GOTM  $k-\epsilon$ model.
More information on the GOTM turbulence closure module can be found in
the GOTM Manual \footnote{http://www.gotm.net/index.php?go=documentation}.

If the turbulence module should be used, a value of |iturb|=2 is recommended.
An example of the settings for the turbulence closure scheme is given in
\Fig\figref{turbulence}.

\begin{figure}[ht]
\begin{verbatim}
$para
        iturb = 2
$end
$name
        gotmpa = 'gotmturb.nml'
$end
\end{verbatim}
\caption{Example of turbulence settings. The GOTM module for the
turbulence closure is used. The parameters are contained in file
gotmturb.nml.}
\label{fig:turbulence}
\end{figure}


