The meshing algorithm is called |mesh|. Please see |mesh -h| for
help of the command line options. 

In order to use the |mesh| algorithm you will have to provide the program
a coastline in which the program will insert triangles. There are some
things to be remembered:

There must be exactly one closed outer (external) line that will enclose
all the other lines given in the coastline file. This means that it is
not possible to mesh two independent domains at the same time. Clearly,
you can divide the grid file into more files, each of which contains
just one independent domain. These files can then be meshed independently.

The program normally is able to find out what is the external line. It
will simply be the line that encloses all the other lines. If no such line
is found, then this will lead to an error. The program will distinguish
between the external line, islands and fault lines. Fault lines are lines
that will constrain triangles to not cross these lines. For example,
putting a fault line along the edge of a channel will ensure that the
triangles will not cross the channel edge but will be placed along
this edge.

In order to decide what line is of which type, the program considers the
largest closest line as the external line, all other lines as islands, and
any open line as a fault line. Normally this is the expected behavior. The
program will classify these lines only if the line type is 0.

If you want to overwrite this behavior you can give explicit line types to
the lines in the coast line file. A type of 1 signals an external line,
a type of 2 an island, and a type of 4 a fault line. Clearly there can
be only one line with type 1. If you have more than one line with type
1 the program will exit with an error. You can however have a fault line
which is closed, a behavior that will not be possible with the automatic
determination of line types, because a closed line with type 0 is always
an island. Clearly an open line with type 2 (island) is also an error.


The |mesh| routine is able to create a grid with uniform or  
different resolution mesh, depending on user interest.

\begin{verbatim}
MESH - Automatic Grid Generation Routine - Version 1.75 
       1995-2009 (c) Georg Umgiesser - ISMAR-CNR        

Options :
  -b   do not refine boundaries        -n   do not insert internal nodes
  -s#  passes for smoothing            -o#  relax. par. for smoothing   
  -I#  number of internal nodes        -B#  number of boundary nodes    
  -R#  overall resolution              -a#  obtain this aspect ratio    
  -g#  element type for background     -h   show this help screen  
\end{verbatim}


If you want a grid with a uniform solution all over, then
you are already in a position to run the meshing algorithm.
You just say: |mesh -I2000 coast-new.grd| and then
the constructed mesh will be in final.grd. The number 2000
means that you want aprox. 2000 internal points in the domain.
You may adjust this number to your needs.

However, you will normally want to have different resolution
in the domain (high at the inlets of lagoons, at interesting
sites like harbours etc..). Then you have to construct a
background grid that gives an indication to the meshing
algorithm what kind of resolution is need in what area.

You open the coastline with grid and construct elements
that cover the parts or all of the domain. The areas where
no background grid exists will use the (constant) resolution of the
domain computed by the routine mesh using the total number of
internal nodes (2000 in this example).

Where a background grid exists the model uses the depth values at the
element vertices (nodes) to compute a new value for this resolution.
The depth value acts like a factor that multiplies the constant
overall resolution to obtain a local resolution. So, for example,
constructing a background grid and setting all depth values to 1
would not change the resolution at all from a situation without
background grid. A factor higher than 1 increases the resolution
and one smaller than 1 decreses it. Therefore, in areas where
resolution should be higher than average you can set it to
2 or higher, and in other areas, where you want lower resolution,
you can set it to 0.5 or lower. All nodes of the background grid
need to have a depth (resolution) value. In side each background
element the resolution is interpolated between the three nodes
(vertices).

In order to distinguish the background grid from the elements
that are constructed by the meshing routine, they must become
a unique element type. You can set it to a value that is not
used for other elements (99 is a good choice). All elements
of the background grid must have this element type.

withse extract the background grid from the grid file you just
have constructed by running exgrd: "exgrd -LS coast-new.grd".
The file "new.grd" contains only the background grid. Rename it to
something more useful (mv new.grd bck1.grd).
The following is a recapitulation of steps for the background grid creation:

\begin{itemize}
\item        manually construct background grid using coast-new.grd
\item        delete coastline (leave only background elements in file)
                |exgrd -LS coast-new.grd|
\item        set depth at nodes for resolution
\item        set type in elements to 99 
\item        rename to bck1.grd
\item        mv new.grd bck1.grd
\end{itemize}


Now you are ready to start the meshing algorithm.

\subsection{Meshing of the basin}
The main options of mesh routine are:\\

        |-I2000|          use aprox. 2000 internal nodes for the domain\\
        |-g99|            element type of background grid is 99\\

With this parameters the call to mesh would be:\\

        |mesh -I2000 -g99 coast-new bck1|\\

The created mesh can be found in final.grd.

Please note that you can specify mor than one file for the coast line,
so you could keep the domain line and the island lines in seperate files.
You can also have different background grid for different areas in
different files. So a call like this is also possible:\\

        |mesh -I2000 -g99 coast island1 island2 bck1 bck2 bck3|\\

After the meshing please have a look at the result (final.grd).
If you need more overall resolution, increase the number of internal
points (here 2000). If you need more resolution in the background grid,
open the background file and increase the factor (depth) value where needed.
You might also need other areas with a background grid. Once you
are satisfied with the result please save it to a more meaningful name.\\

        |mesh -I2000 -g99 coast-new bck1|\\
        |mv final.grd mesh1.grd|\\

\subsection{Adjust elements for regularity}

After the creation of the mesh, the grid is still not good enough
for usage in a finite element model. This is due to the fact that
the grid is too irregular. Therefore a program has to be applied
that regularizes the grid.

The program is called regularize. It must be given the input grid file
(irregular) and creates a new one with much more regular characteristics.
The program has to be called as:\\

        |regularize mesh1.grd mesh2.grd|.\\

In this case the new regular grid is in mesh2.grd.

