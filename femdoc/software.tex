
The source code is composed mainly of Fortran 77 files, but files written
in C, Fortran 90, Perl and Shell scripts are also present.

In order to use the model you have to compile it in a Linux Operating
System. Several software products must be present in order to be able
to compile the model. Please refer to the documentation of your Linux
distribution for installing these programs.

\begin{itemize}

\item A Fortran 77 and 90 compiler. Supported compilers are
the Gnu |g77| and |g90| compilers, the new Gnu compiler |gfortran|,
the Intel Fortran compiler |ifort| or the Portland group Fortran compiler.

\item The package |make| is required for compilation.

\item The |perl| interpreter, the |bash| shell and the |gcc| c compiler
are necessary for compiling.

\end{itemize}

Please note that you might already have everything available in your
Linux distribution, with the exception of the Fortran compiler.

To find out what software is installed on your computer and what you
still have to install you can run the following command:

\begin{code}
    make check_software
\end{code}

If you get something like |bash: make: command not found|, then you
do not have make installed. Please first install the |make| command
before proceeding.

The command will show you what software you will still have to
install. The software is divided into different sections. The first
is needed software, which you will not be able to do without. The next
section is recommended software, which you really should install, but
for compilation and running you will not necessarily need it. The last
section is software which is optional, but which makes life easier.

You can always run |make check_software| again to check if the software
had been successfully installed. When you are satisfied with the output
you can go to the next section.

Please note that you have to carry out the steps in this section only
the first time you install the model. If you install new version of the
software you can skip these steps.




