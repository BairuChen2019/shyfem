
The basic way to run the model is in 2D, computing for each element of the
grid one value for the whole water column.  All the variables are computed
in the center of the layer, halfway down the total depth.  Deeper basins
or highly variable bathymetry can require for the correct reproduction
of the velocities, temperature and salinity the need for 3D computation.

The 3D computation is performed on the basis of z layers. In this
representation each layer horizontally has constant depth over the whole
basin, but vertically the layer thickness may vary between different
layers. However, the first layer (surface layer) is of varying thickness
because of the water level variation, and the last layer of an element
might be only partially present due to the bathymetry.

Layers are counted from the the surface layer (layer 1) down to the
maximum layer, depending again on the local depth. Therefore, elements
(and nodes) normally have a different total number of layers from one to
each other. This is opposed to sigma layers where the number of total
layers is constant all over the basin, but the thickness of each layer
varies between different elements.

In order to use layers for 3D computations a new section |$layers|
has to be introduced into the |STR| file, where the sequence of depth
values of the bottom of the layers has to be declared.  Please, make sure
that in the file |Rules.make|, the number of allowed levels |nlvdim| is
greater or equal than the ones actually used in the |STR| file. Layer
depths must be declared in increasing order. An example of a |$layer|
section is given in figure \figref{layers}. Please note that the maximum
depth of the basin in the example must not exceed 20 m.

\begin{figure}[ht]
\begin{verbatim}
$layers
	2 4 6 8 10 13 16 20
$end
\end{verbatim}
\caption{Example of section {\tt \$layers}. The maximum depth of 
the basin is 20 meters. The first 5 layers have constant thickness 
of 2 m, while the last three vary between 3 and 4 m.}
\label{fig:layers}
\end{figure}

A specific treatment for the bottom layer has to be carried out.  In fact,
if the model runs on basins with variable bathymetry, for each element
there will be a different total number of layers. The bathymetric value
normally does not coincide with one of the layer depths, and therefore
the last layer must be treated seperately.

To declare how to treat the last layer two parameters have to be
inserted in the |$para| section. The first is |hlvmin|, the minimum
depth, expressed as a percentage with respect to the full layer depth,
ranging between 0 and 1, This is the fraction that the last layer
must have in order to be maintained as a seperate layer.  The second
parameter is |ilytyp| and it defines the kind of adjustment done on the
last layer. If it is set to 0 no adjustment is done, if it is set to 1
the depth of the last layer is adjusted to the one declared in the |STR|
file (full layer change).  If it is 2 the adjustment to the previous layer
is done only if the fraction of the last layer is smaller than |hlvmin|
(change of depth).  If it is 3 (default) the bathymetric depth is kept
and added to the last but one layer. Therefore with a value of 0 or 3
the total depth will never be changed, whereas with the other levels
the total depth might be adjusted.

As an example, take the layer definition of \Fig\figref{layers}. Let |hlvmin|
be set to 0.5, and let an element have a depth of 6.5 m. The total number
of layers is 4, where the first 3 have each a thickness of 2 m and the
last layer of this element (layer 4) is 0.5 m. However, the nominal
thickness of layer 4 is 2 m and therefore its relative thickness is 0.25
which is smaller than |hlvmin|. With |ilytyp|=0 no adjustment will be
done and the total number of layers in this element will be 4 and the
last layer will have a thickness of 0.5 m.  With |ilytyp|=1 the total
number of layers will be changed to 3 (all of them with 2 m thickness)
and the total depth will be adjusted to 6 m. The same will happen with
|ilytyp|=2, because the relative thickness in layer 4 is smaller than
|hlvmin|.  Finally, with |ilytyp|=3 the total number of layers will be
changed to 3 but the remaining depth of 0.5 m will be added to layer 3
that will become 2.5 m.

In the case the element has a depth of 7.5 m, the relative thickness is
now 0.75 and greater than |hlvmin|.  In this case, with |ilytyp|=0, 2
and 3 no adjustment will be done and the total number of layers in this
element will be 4 and the last layer will have a thickness of 1.5 m.
With |ilytyp|=1 the total number of layers will be kept as 4 but the
total depth will be adjusted to 8 m. This will make all layers equal to
2 m thickness.

The introduction of layers requires also to define the values of
vertical eddy viscosity and eddy diffusivity.  In any case a value of
these two parameters has to be set if the 3D run is performed. This
could be done by setting a constant value of the parameters |vistur|
(vertical viscosity) and |diftur| (vertical diffusivity). In this case
possible values are between 1\ten{-2} and 1\ten{-5}, depending on the
stability of the water column. Higher values (1\ten{-2}) indicate higher
stability and a stronger barotropic behavior.

The other possibility is to compute the vertical eddy coeficients through
a turbulence closure scheme. This usage will be described in the scetion
on turbulence.

