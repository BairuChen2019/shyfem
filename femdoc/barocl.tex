
The baroclinic pressure gradient term permits to compute
the variation of velocity due to the horizontal
gradients of temperature and/or salinity. These gradients
act on the horizontal variation of density. 

If the variations of temperature and salinity and the
baroclinic pressure gradient term has to be computed, 
the parameter |ibarcl| (in section |$para|)
must be set different from 0.

Setting |ibarcl| to a value different from 0 will simulate the
transport and diffusion of temperature and salinity in the basin. A value of 1
will compute the full baroclinic pressure terms. A value of 2
will do diagnostic simulations. This means that baroclinic pressure terms
are still included in the hydrodynamic equations, 
but temperature and salinity will not be computed
but will be read from a file. Finally for |ibarcl|=3 temperature and salinity
will be computed but no baroclinic pressure term will be used. In this case
the hydrodynamic equations and the equations for temperature and
salinity are decoupled and there is no feed back from the density field
to the currents.

In any case, if temperature and salinity are computed,
first they must be initialized either with
constant values or with variable 3D matrices.
In the first case
the reference values have to be imposed in
|temref| and |salref|. An example of this type of simulation is given
in \Fig\figref{baroc}.

If the temperature and salinity are given as 3D matrices files,
they must be provided in the |$name| section, giving the file 
names in |tempin| and |saltin|. In case of diagnostic simulations the
matrices of temperature and salinity have to be provided in the
files named |tempd| and |saltd| and data must be available for
the whole period of simulation.

\begin{figure}[ht]
\begin{verbatim}
$para
        ibarcl = 1   temref = 18.    salref = 35.
$end
\end{verbatim}
\caption{Example of baroclinic simulation. The initial values for temperature
and salinity are set to 18 C and 35.}
\label{fig:baroc}
\end{figure}

