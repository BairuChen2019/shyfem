
Generally SHYFEM is run with a fixed time step given by the
parameter |idt|.
This choice is acceptable when the model runs in unconditionally
stable conditions (ie. linear simulation, no horizontal viscosity).

The non-linear terms of the momentum advection (|ilin=0|) or 
the horizontal viscosity (|ahpar| greater 0) can introduce computational
instabilities. 
To be sure that the model runs in stable conditions, it must be assured 
that the Courant Number is smaller than 1. 
Please note that only in the case of advection we should call
this number the Courant number. However, we will continue to use
the term Courant number for all stability related issues.

In the case of advection the Courant number is defined as
\begin{equation}
Cou=\frac{v\Delta t}{\Delta x}
\end{equation}
where $v$ is the current speed, $\Delta t$ the time step and $\Delta x$
the element size. For finite elements, due to the triangular grid, this
expression is slightly more complicated. As can be seen, lowering the
time step will bring the Courant number below the limit of 1.

To keep the Courant Number under the limit it is necessary to adapt
the time step at every computation. The variable timestep is computed
introducing in the |str| file in the |$para| section the parameters
|itsplt|, |coumax| and |idtsyn|.

|coumax| gives the limit of the Courant number. This is normally 1,
but since no exact stability limit can be derived for the non-linear
advective terms, another value can be specified. If instabilities arise,
a slightly lower value than 1 (0.9) can be tried.

|itsplt| decides about the time step splitting.  If this value is 0,
the time step will be kept constant at its initial value. A value of 1
devides the initial time step into (possibly) equal parts, but makes sure
that at the end of the micro time steps one complete macro time step has
been executed. The last mode |itsplt| = 2 does not care about the macro
time step, but always uses the biggest time step possible. In this case
it is not assured that after some micro time steps a macro time step
will be recovered. Please note that the initial macro time step |idt|
will never be exceeded.

Finally, the parameter |idtsyn| is only used in case of |itsplt| = 2.
This parameter makes sure that after a time of |idtsyn| the time step
will be syncronized to this time. Therefore, setting |idtsyn| = 3600
means that there will be a time stamp every hour, even if the model has
to take one very small time step in order to reach that time.

An example of how to set the variable time stepping scheme is shown
in \Fig\figref{vartime}. Here the Courant number is lowered to 0.9 and
the variable time step is syncronized every 3600 seconds (1 hour).

\begin{figure}[ht]
\begin{verbatim}
$para
        coumax = 0.9   itsplt = 2   idtsyn = 3600
$end
\end{verbatim}
\caption{Example of variable time step settings. The time step is syncronized
at every hour, and the Courant number is lowered to 0.9.}
\label{fig:vartime}
\end{figure}

