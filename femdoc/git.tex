
%------------------------------------------------------------------------
%
%    Copyright (C) 1985-2018  Georg Umgiesser
%
%    This file is part of SHYFEM.
%
%    SHYFEM is free software: you can redistribute it and/or modify
%    it under the terms of the GNU General Public License as published by
%    the Free Software Foundation, either version 3 of the License, or
%    (at your option) any later version.
%
%    SHYFEM is distributed in the hope that it will be useful,
%    but WITHOUT ANY WARRANTY; without even the implied warranty of
%    MERCHANTABILITY or FITNESS FOR A PARTICULAR PURPOSE. See the
%    GNU General Public License for more details.
%
%    You should have received a copy of the GNU General Public License
%    along with SHYFEM. Please see the file COPYING in the main directory.
%    If not, see <http://www.gnu.org/licenses/>.
%
%    Contributions to this file can be found below in the revision log.
%
%------------------------------------------------------------------------

\subsection{Where to get it}

You can can the latest and older versions of \shyfem{} both as
files on |Google Drive| or on |GitHub|. From |Google Drive| you can
download the latest and older version of the model. Please go to
|https://drive.google.com/open?id=0B742mznAzyDPbGF2em5NMjZYdHc| and
download the version you would like to install on your computer. Normally
this will be the last available version.

You can also download the versions from |GitHub|. Please
go to the |GitHub| website of the \shyfem{} model
|https://github.com/SHYFEM-model/shyfem| and navigate to |releases|. From
there you either can visualize the releases or the tags (versions)
of the model and download them. Please see below for the difference
between tags and releases.

If you are a developer then you should really install the |git|
versioning system that will give you direct access to the latest versions
of \shyfem{}.  Please see below how to do this.

\subsection{Details}

Development of \shyfem{} is happening on |github|. Here is is some
information on how the various releases are are managed and where to
download the latest version.

In github you can always find the latest version of \shyfem{}. There
are various types of versions, which will be explained here below.

\begin{itemize}

\item commit: commits are the smallest change in the code base. Everytime
some changes have been carried out on the code this change will be
commited to the repository. You can see all the commits of the code
by typing |gittags|. This shows you all the commits and also the tags,
which are explained here below.

\item version: the name version is just a shortcut for an existing tag
or release. It has a version number that can be used to refer to a
specific tag or release. A commit has no version number and can therefore not
be identified in this way.

\item tag: tags are like commits, but a version number is given to
them. This means that these tags are more stable than simple commits. It
is always advisable to download tags in order to be able to easily refer
to the version number of \shyfem{}.

\item release: releases are nothing else than tags, but a name is also
given to this tag. This means that releases should be even more stable
than tags or commits. If you do not need a bleeding edge version than
these are the versions that you should download.

\end{itemize}

You can find commits and tags with the command |gittags|. The output of
this command will give you the latest commits and tags (if applied). In
order to see releases you will have to go to the github webpage
(|https://github.com/SHYFEM-model/shyfem|). Click on |releases|, which
takes you directly to page where you can see all the versions, both
as releases or tags. From there you can directly download the latest
version of \shyfem{}.

\subsection{Latest versions and developers}

In order to get also access to the latest single commits (which cannot
be found on the |github| web page) you will have to install |git| on
your computer. Once installed, go to the web page (see address above)
and click on |clone|. This will download the latest version with all
the versioning information included.

Once you have cloned \shyfem{} you can get easy access to the latest
versions and commits by entering the base directory of \shyfem{} and
then type |git fetch| and |git pull|. This will give you the newest
commit of the code base.

If you are a developer you should have git installed on your computer.
If you have worked on a new feature and want your changes being published,
you will have to issue what is called a |pull request|. You can do
this from the |github| website. Please be sure that you base your 
|pull request| on the latest commit (not tag or release). Therefore, once you
are ready with your |pull request|, do a |git fetch| and |git pull|,
check if your changes are compatible, and then do the |pull request|.

All |pull requests| have to be based on the |develop| branch.

